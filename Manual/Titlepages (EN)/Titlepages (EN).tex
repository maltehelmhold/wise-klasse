\documentclass[ trackchanges, hyperref, pdf, mp, mppdf, en]{wise}
%arbeiten mit trackchanges

%Formatierung von Quellcode
\newcommand\code[1]{\texttt{#1}}
\newcommand\command[1]{\code{\textbackslash{}#1}}


\begin{document}
\shorttitlepage{Documentation of English \LaTeX{}-Titlepages \\at the Chair of Information Systems, \\ esp. Systems Engineering}{Effective: \today}{Malte Helmhold}

\begin{preface}
  \tableofcontents
\end{preface}



\section{General}

Before you can use the english Titlepages you must declare the  en-Option in the \command{documentclass}-Command. 





\begin{appendix}\begin{appendices}

\section{Seminartitlepage}

For a \command{seminartitlepage}-command you need 4 parameters (in this order): 

\begin{enumerate}
\item an english Title
\item at least one Name of Student
\item at least one Supervisor
\item the responsible department (english or german) 
\end{enumerate}

Please insert the following code on the correct place (between \command{begin\{document\} } and \command{begin \{preface\}}) of your document:

\lstlatex 

\begin{lstlisting}
\seminartitlepage
{English Title}
{Name of Student 1 (Matr. Nr.)\\
Name of Student 2 (Matr. Nr.)\\
Name of Student 3 (Matr. Nr.) }
{Title and Name of Supervisor 1\\
Title and Name of Supervisor 2\\ 
Title and Name of Supervisor 3 }
{Department}
\end{lstlisting} 
Of course you normally will use ordinary parameters: One Name of Student and just one Supervisor like the following example (please adapt the framed source code and insert it in correct position of your document):
\lstlatex

\begin{lstlisting}
\seminartitlepage
{English Title}
{Name of Student (Matr. Nr.)}
{Name of Supervisor}
{Department}

\end{lstlisting}




\clearpage

\seminartitlepage
{English Title}
{Name of Student 1 (Matr. Nr.)\\
Name of Student 2 (Matr. Nr.)\\
Name of Student 3 (Matr. Nr.) }
{Name of Supervisor 1\\
Name of Supervisor 2\\ 
Name of Supervisor 3}
{Department}

\section{Doubleseminartitlepage}
For a \command{doubleseminartitlepage}-command you need 6 parameters (in this order):
\begin{enumerate}
\item An english title
\item A german title
\item At least one name of student
\item At least one supervisor
\item The responsible department (german name)
\item The responsible department (english name)
\end{enumerate}

\textbf{Please insert the framed sourcecode into your document  in order to  get an english/german seminarTitlepage. You can see an example on follogwing two pages. }\\
\lstlatex
\begin{lstlisting}
\doubleseminartitlepage
{English Title}
{German Title}
{Name of Student 1 (Matr. Nr.) \\ 
Name of Student 2 (Matr. Nr.) \\ 
Name of Student 3 (Matr. Nr)}
{Title and Name of Supervisor 1 \\ 
Title and Name of Supervisor 2}
{Department (for english titlepage)}
{Fakult�t (Department) for german titlepage)}
\end{lstlisting}
Please note the output for this source code on following page. \\

Of course you normally will use ordinary parameters: One Name of Student and just one Supervisor like the following example (please adapt the framed source code and insert in correct position of your document):

\lstlatex
\begin{lstlisting}
\doubleseminarTitlepage
{English Title}
{German Title}
{Name of Student 1 (Matr. Nr.)}
{Title and Name of Supervisor}
{Department (for english Titlepage)} 
{Fakult�t (Department) for german Titlepage)}
\end{lstlisting}

\clearpage

\doubleseminartitlepage
{English Title}
{German Title}
{Name of Student 1 (Matr. Nr.) \\ Name of Student 2 (Matr. Nr.) \\ Name of Student 3 (Matr. Nr)}
{Title and Name of Supervisor 1 \\ Title and Name of Supervisor 2}
{Department}{Fakult�t}

\section{Projecttitlepage}

For a \command{seminartitlepage}-command you need 4 parameters (in this order): 

\begin{enumerate}
\item An english title
\item At least one name of student
\item At least one supervisor
\item The responsible department (english or german) 
\end{enumerate}

Please insert the following code on the correct place (between \command{begin\{document\} } and \command{begin \{preface\}}) of your document:

\lstlatex 

\begin{lstlisting}
\projecttitlepage
{English Title}
{Name of Student 1 (Matr. Nr.)\\
Name of Student 2 (Matr. Nr.)\\
Name of Student 3 (Matr. Nr.) }
{Name of Supervisor 1\\
Name of Supervisor 2\\ 
Name of Supervisor 3 }
{Department}
\end{lstlisting} 
Please note the output for this source code on following page. \\

Of course you normally will use ordinary parameters: One Name of Student and just one Supervisor like the following example (please adapt the framed source code and insert it in correct position of your document):
\lstlatex
\begin{lstlisting}
\projecttitlepage
{English Title}
{Name of Student (Matr. Nr.)}
{Name of Supervisor}
{Department}

\end{lstlisting}




\clearpage 

\projecttitlepage
{English Title}
{Name of Student 1 (Matr. Nr.)\\
Name of Student 2 (Matr. Nr.)\\
Name of Student 3 (Matr. Nr.) }
{Name of Supervisor 1\\
Name of Supervisor 2\\ 
Name of Supervisor 3 }
{Department}

\section{Bachelortitlepage}\label{sec:titelblatt_bachelorarbeiten}

For a \command{bachelortitlepage}-command you need 8 parameters (in this order): 

\begin{enumerate}
\item An english title
\item Degree
\item Name of student
\item Student number
\item Title and name of 1st supervisor
\item Title and name of 2nd supervisor
\item Beginning date of working period
\item Termination date of working period
\end{enumerate}

Please insert the following code on the correct place (between \command{begin\{document\} } and \command{begin \{preface\}}) of your document:

\lstlatex 

\begin{lstlisting}
\bachelortitlepage
{Title}
{Bachelor of Science}
{Name of Student}
{1234567}
{Title and Name of 1st  Supervisor}
{Title and Name of 2nd  Supervisor}
{01.01.2040}
{30.06.2040}
\end{lstlisting} 

Please note the output for this source code on following page.



\clearpage

\bachelortitlepage{Title}{Bachelor of Science}{Name of Student}{1234567}{Title and Name of 1st  Supervisor}{Title and Name of 2nd  Supervisor}{01.01.2040}{30.06.2040}

\section{Mastertitlepage}\label{sec:titelblatt_masterarbeiten}

For a \command{mastertitlepage}-command you need 8 parameters (in this order): 

\begin{enumerate}
\item An english title
\item Degree
\item Name of student
\item Student number
\item Title and name of 1st supervisor
\item Title and name of 2nd supervisor
\item Beginning date of working period
\item Termination dat of working period
\end{enumerate}

Please insert the following code on the correct place (between \command{begin\{document\} } and \command{begin \{preface\}}) of your document:

\lstlatex 

\begin{lstlisting}
\bachelortitlepage
{Title}
{Bachelor of Science}
{Name of Student}
{1234567}
{Title and Name of 1st  Supervisor}
{Title and Name of 2nd  Supervisor}
{01.01.2040}
{30.06.2040}
\end{lstlisting} 

Please note the output for this source code on following page. 



\clearpage

\mastertitlepage{Title}{Master of Science}{Name of Student}{1234567}{Title and Name of 1st Supervisor}{Title and Name of 2nd Supervisor}{01.01.2040}{30.06.2040}



\end{appendices}\end{appendix}
\end{document}
